\documentclass[12pt]{article}
\bibliography{ribozymes}
\bibliographystyle{nature}

\usepackage[english]{babel} 

\begin{document}
\begin{titlepage}
\title{Acid and base catalysis in catalytically active ribonucleic acid sequences}
\author{Malin L\"{u}king}

\end{titlepage}
\maketitle
 \pagenumbering{gobble}
  \newpage
   \pagenumbering{arabic}
   	
\begin{abstract}

\end{abstract}

\section{project plan}
\subsection{Introduction}
The here proposed project is aiming mapping out possibilities in the combination of two trending fields in today's biochemistry. The proposed project is designed to explore the field of ribozymes using molecular dynamics. Todays computational techniques make it possible to model the dynamics and evolutionary development of large molecules like enzymes. Helping to reveal reaction mechanism and transition state structures in the active sites supports the the rational design of biocatalysts with addiotinal information.
RNA is an interesting biomolecule because of several resasons. More and more knowledge is gained when it comes to its role in the regulatory mechanism in the cells differentiation and specialization. Deeper investigations of RNA metabolism and their diverse structures with different roles in cells revealed their ability to be self-catalytic. This gave rise to the idea of the origin of life built from RNA rather than proteins. In this sense using the catalytic functionality of RNA is a field that should be considered to be deeper understood for the design of catalysts.
So far some ribozymes and their catalytic mechanism have been studied using MD simulations. What makes them especially interesting is that this mechnism is mainly based on general acid and base catalysis, which is one of the main reaction mechanism taking place in biology. Transfering reactions interesting reactions to RNA based systems is therefore interesting to consider. As it was first used in nature it might also be a easier system to use in de novo design of biocatalysts. 

\subsection{Motivation and current disscussion}
The advantage of RNA is that it is considting of only 4 building blocks, the nucleobases adenosin, guanosin, uridin and cytidin.
\subsubsection{collaborations} 
The files and structural data will be provided by our collaborators. The project will be carried out by a post doctoral researcher and also provide work wor one PhD student. The project is supposed yoield the first results after 5 years of work which would imply development of appropriate force fields for ribozymes, a data base, that provides for structurel features that have been ididentified as catalytic active and promising for engineering new catalysts based on RNA. Also the link between function and sequence should be further elucidated.Interesting for design of catalysts based on RNA structures is the development of techniques allowing to calculate back to the catalytic RNA which is carrieing out functions without being in a large ensemble with proteins, like it is the case for proteins. It was proposed that the actual cataalytic function is stored in the RNA not in the protein and that the origin in life wís based on catalytic RNA which was first linking a sequence to a function.

--> read into RNA world

--> collaborations with groups working on RNA and on RNA force fiels in Stockholm and Uppsala
--> evolutionary backtracing of reactivity stored in RNA alone 
\subsection{Methodes}
free energie calculations 

\subsubsection{Ribozymes}

\subsection{Acid & Base catalysis}

\subsection{Budget}

\subsection{Timeplan}

\subsection{•}

\end{document}

